\documentclass[11pt]{article}
\usepackage{graphicx}
\graphicspath{
        {../img}
}
\newcommand{\rta}{\\}
\usepackage [spanish]{babel}
\title{
    \includegraphics[width=0.25\textwidth]{unpsjb_notext.png} \\
    \vspace{1.5cm}
    Señales y Sistemas \\ 
    Trabajo Práctico 2
    }
\date {Primer Cuatrimestre 2025}
\author{Alan Gustavo Bitsch}



\begin{document}
\maketitle



\section{Señales y Sistemas: conceptos básicos}
Explique con sus propias palabras las definiciones de
\begin{enumerate}
    \item Señal:
    \rta Una señal es una función que nos describe una magnitud de interés de un sistema.
    
    \item Variable dependiente e independiente:
    \rta Una variable independiente es aquella que no depende de ninguna otra variable. En contraste, una variable dependiente es aquella que depende de una o más variables en su definición.
    
    \item Tiempo continuo y tiempo discreto:
    \rta El tiempo discreto es aquel que está discretizado, por lo que avanza de a pasos y solo puede tomar una cantidad finita de valores para un intervalo finito.
    \rta El tiempo continuo, al no estar discretizado, es infinitamente divisible en conjuntos más pequeños. Un tiempo continuo en un intervalo finito puede tomar infinitos valores.
    
    \item Muestreo:
    \rta Se refiere al proceso de tomar una señal de tiempo continuo y(t) y convertirla en una señal de tiempo discreto y[t] al evaluarla en intervalos de tiempo regulares. 
    
    \item Cuantización:
    \rta Es el proceso de tomar el valor de tiempo de cada medición y truncarlo o redondearlo, para que y[t] quede definida para números naturales.   
    
    \item Entradas y salidas de un sistema (también llamada interface)
    \rta Las entradas y salidas de un sistema son señales. Las entradas, al ser procesadas, se convierten en la salida del sistema; Por esto podemos decir que las entradas son aquellas que estimulan el sistema, y las salidas son la respuesta del sistema al estímulo.
    \rta Definimos interfaz de entrada y de salida al conjunto de señales (de entrada y de salida) de un sistema.
    
    \item Sistemas estáticos y dinámicos:
    \rta Decimos que un sistema es estático (o sin memoria) cuando la salida del mismo para una entrada es independiente de los valores pasados o futuros de entrada. La salida permanece igual dada una misma entrada.
    \rta Por el contrario, un sistema dinámico es aquel que sí tiene memoria: su evolución depende de las entradas previas o posteriores, por lo que su salida puede variar para la misma entrada.

    \item sistemas lineales:
    \rta Los sistemas lineales son aquellos que cumplen con la condición de linealidad: 
        f es lineal si $f(\alpha x_1 + \beta x_2)
    =\alpha f(x_1) + \beta f(x_2)$
    \item sistemas invariantes con el tiempo:
    \rta Los sistemas invariantes con el tiempo son aquellos cuyo comportamiento no es dependiente del tiempo o, lo que es lo mismo,
        \\ $f(t) = f(t+t_0)$ para todo $t_0$

\end{enumerate}



\end{document}